%#!pdfplatex
%
% How to make report01.pdf
% % latex report01.tex
% % latex report01.tex
% % dvipdf report01.dvi
% 
\documentclass{article}
\title{Programming Language Processor \\ Assignment 2}
\author{Your ID   Your name}
\date{\today}

\def\reporttrue{\let\ifreport=\iftrue}
\def\reportfalse{\let\ifreport=\iffalse}
%\reportfalse  % question
\reporttrue  % report



\begin{document}
\ifreport
\maketitle
\else
\begin{center}
{\huge Programming Language Processor Assignment 2}
\end{center}
Answer the following questions and submit your report (Word or PDF) to
tetsuya@shibaura-it.ac.jp before Dec. 15. 
The subject of your mail should be of the form ``PLP Assignment 2''.
\fi


%%%%%%%%%%%%%%%%%%%%%%%%%%%%%%%%%%%%%%%%%%%%%%%%%%%%%%%%%%%%%%%%%%%%%%%%%%%%%%%%%%%%%%%%%%%%%%%
\section*{Question 1}
Modify the sytax diagram of PL/0' as follows, and explain the modified parts of the diagram.
\begin{itemize}
 \item Introduce the following do-while statement to PL/0'.
 \item Introduce the following repeat-until statement to PL/0'.
 \item Introduce the following if-then-else statement to PL/0'.
\end{itemize}


\subsection*{do-while statement}
Introduce the following do-while statement into PL/0'.
\begin{description}
 \item[Production rule] {\it statement}  $\to$ {\bf do} {\it statement} {\bf while} {\it condition}
 \item[Action] A statement '{\bf do} {\it statement} {\bf while} {\it condition}' works as follows
	    \begin{enumerate}
	     \item Execute {\it statement}.
	     \item If the value of {\it condition} is true, go to the step 1. Otherwise, exit this loop.
	    \end{enumerate}
\end{description}

Fig.\ref{fig:do-while} shows a sample program with a do-while statement.
\begin{figure}[h]
\begin{verbatim}
var x;
begin
   x := 0;
   do begin
      write x;
      writeln;
      x := x + 1
   end
   while x < 3
end.
\end{verbatim}
\caption{A sample program {\tt do.pl0}}\label{fig:do-while}
\end{figure}

\subsection*{repeat-until statement}
Introduce the following repeat-until statement into PL/0'.
\begin{description}
 \item[Production rule] {\it statement}  $\to$ {\bf repeat} {\it statement} {\bf until} {\it condition}
 \item[Action] A statement '{\bf repeat} {\it statement} {\bf until} {\it condition}' works as follows.
	    \begin{enumerate}
	     \item Execute {\it statement}.
	     \item If the value of {\it condition} is false, go to the step 1. Otherwise, exit this loop.
	    \end{enumerate}
\end{description}

Fig.\ref{fig:repeat-until} shows a sample program with a repeat-until statement.
\begin{figure}[h]
\begin{verbatim}
var x;
begin
   x := 0;
   repeat begin
      write x; 
      writeln;
      x := x + 1
   end
   until x=3
end.
\end{verbatim}
\caption{A sample program {\tt repeat.pl0}}\label{fig:repeat-until}
\end{figure}

\subsection*{if-then-else statement}
Modify the syntax diagram so that PL/0' can accept the following if-then-else statement.
\begin{description}
 \item[Production rule] {\it statement}  $\to$ {\bf if} {\it condition} {\bf then} 
       ${\it statement}_1$ ({\bf else} ${\it statement}_2$ $\vert$ {$\epsilon$})
 \item[Action] A statement '{\bf if} {\it condition} {\bf then} 
	    ${\it statement}_1$ ({\bf else} ${\it statement}_2$ $\vert$ {$\epsilon$})'
	    works as follows.
	    \begin{enumerate}
	     \item Evaluate {\it condition}.
	     \item If the value of {\it condition} is true, execute ${\it statement}_1$.
	     \item If the value of {\it condition} is false and ${\it statement}_2$ exists, 
		   execute ${\it statement}_2$.
	    \end{enumerate}
 \item[Description] To resolve ambiguity of the grammar of PL/0', 
	    we use the following rule.
	    \begin{itemize}
	     \item When we find an {\bf else}, we relate the {\bf else} to the nearest {\bf then}
		   which has not be related to any {\bf else} yet.
	    \end{itemize}
\end{description}

Fig.\ref{fig:if-then-else} shows a sample program with if-then-else statements.
\begin{figure}[h]
\begin{verbatim}
var x;
begin
   x := 0;
   while x<3 do begin
      if x < 1 then write 0
      else if x < 2 then write 1
      else write 2;
      writeln;
      x := x+1;
   end;
end.
\end{verbatim}
\caption{A sample program {\tt else.pl0}}\label{fig:if-then-else}
\end{figure}


\ifreport
(Answer)\\
\fi
% Write your answer.




%%%%%%%%%%%%%%%%%%%%%%%%%%%%%%%%%%%%%%%%%%%%%%%%%%%%%%%%%%%%%%%%%%%%%%%%%%%%%%%%%%%%%%%%%%%%%%%
%\newpage
\section*{Question 2}
Modify your syntax diagram in Question 1 so that PL/0' can accept one-dimensional array,
and explain the modified parts of the diagram.
In addition, write a simple program {\ tt array.pl0} which demonstrates how to use your array.

\begin{description}
 \item[Hints] You have to consider how to declare arrays, 
	    how to refer to array elements and how to assign values to array elements.
\end{description}

\ifreport
(Answer)\\
\fi
% Write your answer.





%%%%%%%%%%%%%%%%%%%%%%%%%%%%%%%%%%%%%%%%%%%%%%%%%%%%%%%%%%%%%%%%%%%%%%%%%%%%%%%%%%%%%%%%%%%%%%%
\section*{Question 3}

Modify your syntax diagram of PL/0' in Question 1 so that PL/0' 
can accept procedure declarations and procedure calls.
A procedure means a function without any return value like a void function in C.

We use the following statement to call a procedure with $n$ arguments.
\begin{quote}
 {\bf call} {\it procedure}($arg_1$, $arg_2$, $\dots$, $arg_n$)
\end{quote}

Explain the modified parts of your syntax diagram and 
write a simple program 'proc.pl0' which demonstrates how to declare and call procedures.


\ifreport
(Answer)\\
\fi
% Write your answer.




\end{document}