%#!pdfplatex
%
% How to make report01.pdf
% % latex report01.tex
% % latex report01.tex
% % dvipdf report01.dvi
% 
\documentclass{article}
\title{Programming Language Processor \\ Assignment 3}
\author{Your ID   Your name}
\date{\today}

\def\reporttrue{\let\ifreport=\iftrue}
\def\reportfalse{\let\ifreport=\iffalse}
%\reportfalse  % question
\reporttrue  % report


\begin{document}
\ifreport
\maketitle
\else
\begin{center}
{\huge Programming Language Processor Assignment 3}
\end{center}
Answer the following questions and submit your report (Word or PDF) to
tetsuya@shibaura-it.ac.jp before Jan. 19, 2015. 
The subject of your mail should be of the form ``PLP Assignment 3''.
\fi


%%%%%%%%%%%%%%%%%%%%%%%%%%%%%%%%%%%%%%%%%%%%%%%%%%%%%%%%%%%%%%%%%%%%%%%%%%%%%%%%%%%%%%%%%%%%%%%
\section*{Question 1}

To introduce the following do-while statement to PL/0', answer the following questions.
\begin{description}
 \item[Production rule] {\it statement}  $\to$ {\bf do} {\it statement} {\bf while} {\it condition}
 \item[Action] A statement '{\bf do} {\it statement} {\bf while} {\it condition}' works as follows
	    \begin{enumerate}
	     \item Execute {\it statement}.
	     \item If the value of {\it condition} is true, go to the step 1. Otherwise, exit this loop.
	    \end{enumerate}
\end{description}

\subsection*{Question 1-1}
To add a token {\tt do} to a set of starting tokens of {\it statement},
modify a function {\tt isStBeginKey} in {\tt compile.c} and explain the modification in your report.

\ifreport
(Answer)\\
\fi
% Write your answer.



\subsection*{Question 1-2}
Modify a function {\tt statement} in {\tt compile.c} 
so that your PL/0' compiler can output object codes of Fig.\ref{fig:dowhile-code} for
do-while statements.
Explain the modificaiton in your report.

\begin{figure}[h]
\begin{tabular}{ll}
label1: & Object codes of {\it statement} \\
        & Object codes of {\it condition} \\
        & jpc label2 \\
        & jmp label1 \\
label2: & \\
\end{tabular}
\caption{Object codes for a do-while statement}
\label{fig:dowhile-code}
\end{figure}

\ifreport
(Answer)\\
\fi
% Write your answer




\subsection*{Question 1-3}
What does your PL/0' compiler outputs when your PL/0' compiler compiles
and executes a PL/0' program do.pl0 of Fig. \ref{fig:do-while}?

\begin{figure}[h]
\begin{verbatim}
var x;
begin
   x := 0;
   do begin
      write x;
      writeln;
      x := x + 1
   end
   while x < 3
end.
\end{verbatim}
\caption{A test program {\tt do.pl0}}\label{fig:do-while}
\end{figure}


\ifreport
(Answer)\\
\fi
% Write your answer.





%%%%%%%%%%%%%%%%%%%%%%%%%%%%%%%%%%%%%%%%%%%%%%%%%%%%%%%%%%%%%%%%%%%%%%%%%%%%%%%%%%%%%%%%%%%%%%%
\newpage
\section*{Question 2}
Answer the following questions to add the following repeat-until statement to PL/0'.

\begin{description}
 \item[Production rule] {\it statement}  $\to$ {\bf repeat} {\it statement} {\bf until} {\it condition}
 \item[Action] A statement '{\bf repeat} {\it statement} {\bf until} {\it condition}' works as follows.
	    \begin{enumerate}
	     \item Execute {\it statement}.
	     \item If the value of {\it condition} is false, go to the step 1. Otherwise, exit this loop.
	    \end{enumerate}
\end{description}

\subsection*{Question 2-1}
Write object codes for the repeat-until statement 
like object codes for the do-while statement of Fig.\ref{fig:dowhile-code}.

\ifreport
(Answer)\\
\fi
% Write your answer




\subsection*{Question 2-2}
Modify {\tt getSource.h} and {\tt getSource.c} to register two tokens
{\tt repeat} and {\tt until} to your PL/0' compiler.
Explain the modification in your report.

\ifreport
(Answer)\\
\fi
% Write your answer.



\subsection*{Question 2-3}
To add a token {\tt repeat} to a set of starting tokens of {\it statement},
modify a function {\tt isStBeginKey} in {\tt compile.c} and explain the modification in you report.

\ifreport
(Answer)\\
\fi
% Write your answer.



\subsection*{Question 2-4}
Modify a function {\tt statement} in {\tt compile.c}
so that your PL/0' compiler can output object codes for repeat-until statements.
Explain the modificaiton in your report.

\ifreport
(Answer)\\
\fi
% Write your answer.




\subsection*{Question 2-5}
What does your PL/0' compiler outputs when your PL/0' compiler compiles
and executes a PL/0' program {\tt repeat.pl0} of Fig.\ref{fig:repeat-until}?

\begin{figure}[h]
\begin{verbatim}
var x;
begin
   x := 0;
   repeat begin
      write x; 
      writeln;
      x := x + 1
   end
   until x=3
end.
\end{verbatim}
\caption{A test program {\tt repeat.pl0}}\label{fig:repeat-until}
\end{figure}


\ifreport
(Answer)\\
\fi
% Write your answer.




%%%%%%%%%%%%%%%%%%%%%%%%%%%%%%%%%%%%%%%%%%%%%%%%%%%%%%%%%%%%%%%%%%%%%%%%%%%%%%%%%%%%%%%%%%%%%%%
\newpage
\section*{Question 3}
Answer the following questions to add the following if-then-else statement to PL/0'.

\begin{description}
 \item[Production rule] {\it statement}  $\to$ {\bf if} {\it condition} {\bf then} 
       ${\it statement}_1$ ({\bf else} ${\it statement}_2$ $\vert$ {$\epsilon$})
 \item[Action] A statement '{\bf if} {\it condition} {\bf then} 
	    ${\it statement}_1$ ({\bf else} ${\it statement}_2$ $\vert$ {$\epsilon$})'
	    works as follows.
	    \begin{enumerate}
	     \item Evaluate {\it condition}.
	     \item If the value of {\it condition} is true, execute ${\it statement}_1$.
	     \item If the value of {\it condition} is false and ${\it statement}_2$ exists, 
		   execute ${\it statement}_2$.
	    \end{enumerate}
 \item[Description] To resolve ambiguity of the grammar of PL/0', 
	    we use the following rule.
	    \begin{itemize}
	     \item When we find an {\bf else}, we relate the {\bf else} to the nearest {\bf then}
		   which has not be related to any {\bf else} yet.
	    \end{itemize}
\end{description}


\subsection*{Question 3-1}
Write object codes for 
a statement '{\bf if} {\it condition} {\bf then}
${\it statement}_1$ {\bf else} ${\it statement}_2$'
like object codes for a do-while statement of Fig.\ref{fig:dowhile-code}.

\ifreport
(Answer)\\
\fi
% Write your answer.




\subsection*{Question 3-2}
Modify {\tt getSource.h} and {\tt getSource.c} to register a token
{\tt else} to your PL/0' compiler.
Explain the modification in your report.

\ifreport
(Answer)\\
\fi
% Write your answer.




\subsection*{Question 3-3}
Modify a function {\tt statement} in {\tt compile.c}
so that your PL/0' compiler can output object codes for if-then-else statements.
Explain the modificaiton in your report.

\ifreport
(Answer)\\
\fi
% Write your answer.




\subsection*{Question 3-4}
What does your PL/0' compiler outputs when your PL/0' compiler compiles
and executes a PL/0' program else.pl0 of Fig.\ref{fig:if-then-else}?

\begin{figure}[h]
\begin{verbatim}
var x;
begin
   x := 0;
   while x<3 do begin
      if x < 1 then write 0
      else if x < 2 then write 1
      else write 2;
      writeln;
      x := x+1;
   end;
end.
\end{verbatim}
\caption{A test program else.pl0}\label{fig:if-then-else}
\end{figure}


\ifreport
(Answer)\\
\fi
% Write your answer.



%%%%%%%%%%%%%%%%%%%%%%%%%%%%%%%%%%%%%%%%%%%%%%%%%%%%%%%%%%%%%%%%%%%%%%%%%%%%%%%%%%%%%%%%%%%%%%%
\newpage
\section*{Question 4}
Answer the following questions to introduce one-dimensional array to PL/0'.


\subsection*{Question 4-1}
Explain how to modify the grammar of PL/0' to introduce one-dimensional array to PL/0'.

\ifreport
(Answer)\\
\fi
% Write your answer.



\subsection*{Question 4-2}
Do you need new instructions to the PL/0' virtual machine for one-dimensional array?
If you need new instructions, define thier mnemonics and their actions. 

\ifreport
(Answer)\\
\fi
% Write your answer.




\subsection*{Question 4-3}
Modify your PL/0' compiler so that it can support one-dimensional array.
Explain the modification in your report.

\ifreport
(Answer)\\
\fi
% Write your answer.




\subsection*{Question 4-4}
Write a simple test program {\tt array.pl0} for one-dimensional array.
Explain the test program and what your PL/0' compiler outputs when it
compiles and executes the test program.

\ifreport
(Answer)\\
\fi
% Write your answer.




%%%%%%%%%%%%%%%%%%%%%%%%%%%%%%%%%%%%%%%%%%%%%%%%%%%%%%%%%%%%%%%%%%%%%%%%%%%%%%%%%%%%%%%%%%%%%%%
\newpage
\section*{Question 5}
Answer the following questions to introduce procedures (functions withaout any return values) to PL/0'.

We use the following statement to call a procedure with $n$ arguments.
\begin{quote}
 {\bf call} {\it procedure}($arg_1$, $arg_2$, $\dots$, $arg_n$)
\end{quote}


\subsection*{Question 5-1}
Explain how to modify the grammar of PL/0' to introduce procedures to PL/0'.

\subsection*{Question 5-2}
Do you need new instructions to the PL/0' virtual machine for procedures?
If you need new instructions, define thier mnemonics and their actions. 


\subsection*{Question 5-3}
Modify your PL/0' compiler so that it can support procedures.
Explain the modification in your report.

\ifreport
(Answer)\\
\fi
% Write your answer.


\subsection*{Question 5-4}
Write a simple test program {\tt proc.pl0} for procedures.
Explain the test program and what your PL/0' compiler outputs when it
compiles and executes the test program.


\ifreport
(Answer)\\
\fi
% Write your answer.



%%%%%%%%%%%%%%%%%%%%%%%%%%%%%%%%%%%%%%%%%%%%%%%%%%%%%%%%%%%%%%%%%%%%%%%%%%%%%%%%%%%%%%%%%%%%%%%
\newpage
\section*{Question 6}
Introduce your own idea to your PL/0' compiler.

\ifreport
(Answer)\\
\fi
% Write your answer.



%%%%%%%%%%%%%%%%%%%%%%%%%%%%%%%%%%%%%%%%%%%%%%%%%%%%%%%%%%%%%%%%%%%%%%%%%%%%%%%%%%%%%%%%%%%%%%%
%\vspace{1cm}
\ifreport
\else
\section*{How to submit your report}
Submit an archive file which includes the following files to
tetsuya@shibaura-it.ac.jp before Jan. 19, 2015.
\begin{enumerate}
 \item Your report file report03.pdf or report03.doc (Word)
 \item All source files of your PL/0' compiler
 \item Makefile
 \item The following test programs
       \begin{enumerate}
	\item array.pl0
	\item proc.pl0
	\item test programs for your own idea
       \end{enumerate}
\end{enumerate}

The subject of your mail should be of the form ``PLP Assignment 3''.

A name of the archive file should be of the form ``your\_id.tgz'' like ``xa14000.tgz''.
You can make the the archive file on Linux as follows.
\begin{enumerate}
 \item Create a directory whose name is your ID.
       For example, create a directory ``xa14000'' as follows if your ID is ``xa14000''.
       \begin{quote}
	{\tt \% mkdir} {\tt xa14000} 
       \end{quote}

 \item Copy all files into the directory.
 \item Make your archive file using tar command as follows.
       \begin{quote}
	{\tt \% tar zcvf} \ \ {\tt xa14000}{\tt .tgz} \ \ {\tt xa14000}
       \end{quote}
\end{enumerate}

Check your archive file before submission. You can expand your archive
file as follows.
\begin{quote}
 {\tt \% tar zxvf} \ \ {\tt xa14000}{\tt .tgz}
\end{quote}


\fi

\end{document}